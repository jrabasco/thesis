%\begingroup
%\let\cleardoublepage\clearpage


% English abstract
\cleardoublepage
\chapter*{Abstract}
%\markboth{Abstract}{Abstract}
\addcontentsline{toc}{chapter}{Abstract} % adds an entry to the table of contents
% put your text here
The technology we have access to nowadays shapes the way we live our lives and our social interactions. This has an effect on our memory and the way we memorize the facts and events of our everyday life. Social media platforms have changed the way we interact with others and share information with our friends. As a matter of fact, Internet users are exposed daily to a staggering amount of personal information, especially since those Online Social Networks started to play a key role in their lives (i.e., smartphone users spend 20\% of their time on the Facebook application).\\
A great deal of effort has been spent in recommending relevant content to the user (e.g., targeted advertisements, similar news, etc.) as it is really useful both for both content providers and users. However, there are no comprehensive studies on how much of this information is memorized by the user.\\
Reminisce.me is a platform which can help study this exact aspect. It provides a way to collect information about a user on Facebook and then ask them questions about it. Rather than answering tedious surveys, it  a fun and entertaining alternative.\\
In this project we improved Reminisce.me by giving it the ability to generate questions with a adaptable difficulty. This aims at making the game more challenging and interesting to play. The focus of this thesis is to explain the assumptions that were made about the memorability of the information we gather from Facebook, how those assumptions shaped the the creation of questions with adaptable difficulty and the impact it had on the performance of the users.
%\endgroup			
%\vfill
