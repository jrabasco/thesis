%\begingroup
%\let\cleardoublepage\clearpage


% English abstract
\cleardoublepage
\chapter*{Abstract}
%\markboth{Abstract}{Abstract}
\addcontentsline{toc}{chapter}{Abstract} % adds an entry to the table of contents
% put your text here
The technology we have access to nowadays shapes the way we live our lives and the way we interact with others. This has an effect on our memory and the way we memorize the facts and events of our everyday life. Social media platforms have changed the way we interact with others and share information with our friends. As a matter of fact, Internet users are exposed daily to a staggering amount of personal information, especially since Online Social Networks started to play a key role in their lives (i.e., smartphone users spend 20\% of their time on the Facebook app).\\
A great deal of effort has been spent in recommending relevant content to the user (e.g., targeted advertisements, similar news, etc.) as it is really useful both for both content providers and users. However, there are no comprehensive studies on how much of this information is memorised by the user.\\
In this project we created a platform which will help study this exact aspect. We provide a way to collect information about a user on Facebook and then ask them question about it. Rather than answering tedious surveys, we propose a fun and entertaining alternative: we let the user play a game where answering questions about their own Facebook activity correctly is key to winning. The questions use different categories of memory triggers which will allow us to identify which kind of activity posted on a social network are more memorable.\\
We found out that people remember more about their interaction with others (which friend commented on a post, who liked some post) than the precise date of the interaction. We also found that our subjects tend to not share a lot about their location but they tend to remember it when they do.


%\endgroup			
%\vfill
