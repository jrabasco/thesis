\chapter{User Comfort Improvement}
One of the main goals of ReminisceMe is to get as many information from the players. It is crucial that the users play as many games as possible and that they keep playing for as long as possible. The more a person plays the more detailed the analysis can be. It is also important that the users enjoy the game enough for them to talk about it and attract more users. For that to be possible, we need the game to be enjoyable and the player to feel comfortable while playing. A lot of measures have been been put in place to make the whole experiment as good as possible.
\section{Mobile improvements}
The biggest part of Facebook users are mobile users \cite{mobileusage}. Therefore, we expect a decent portion of our user base to be mobile users, playing while on transportation or during a break. Making the experience great for these players is then an important concern.
\subsubsection{Mobile application}
The main challenges to have a smooth mobile experience are design issues and bandwidth limitations. One of the best ways to overcome these difficulties is to provide the user with a standalone mobile application. This both helps with serving a nice responsive mobile layout and allows the user to download the application before using it, this is helpful as the desktop version is quite heavy and is not really friendly with the narrow badnwidth mobile data offers.
\subsubsection{No post ordering}
Despite having a responsive design on a mobile application, certain elements cannot be displayed properly and even look weird on a large screen. The main difficulty about displaying posts is that their length and content cannot predicted and some large posts (which are consequently also more interesting) cannot be rendered nicely inside an ordering item. The decision was therefore made to drop all the questions which would ask to reorder posts.
\section{Blacklist}
While the application was being tested by different users, some users came across content related to people they did not wish to hear about. While this is a really unfortunate situation and it might diminish our capacity to generate questions, we felt compelled to add the possibility for the user to ignore users they do not want any mention of on their game boards. The main reason behind this decision is that, depending on the relationship between a potential player and some of the people mentioned in the questions, the discomfort can be big enough for the user to not want to play the game again. Unfortunately, it is impossible for us to predict when such a situation might arise and we cannot do anything preemptively [HIGHLIGHT BLACKLIST IN INTERFACE?], but we hope that having the ability to ignore people is sufficiant to make users stay after encountering such a situation.
\section{Notifications}