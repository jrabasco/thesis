\chapter{Questions difficulty}
During the first public release of Reminisce.me, family and friends were asked to play the game and provide feedback. With the help of the 58 games played, we were able to get information about some important aspects of the game: which questions are selected by the users, are they answering randomly or by thinking, how much time they take to answer the questions. This feedback was really useful to understand how difficult the game was and what kind of improvement has to be made. One of the first issue was that the questions were generated completely randomly and there was no concern about how difficult they were, we mostly used parameters that we thought would make sense and not too difficult. We had two main reasons for wanting to be able to change the difficulty at will.\\
The first and most obvious reason is that it makes the game more enjoyable, getting hard questions and failing over and over is  not amusing and the same goes if you only get easy question which offer no challenge at all. The second reason is that by being able to toy with the difficulty, we can test hypotheses and improve our understanding of memory: if we think that some parameter is a central element of what makes an element memorable we can then consider it when modifying questions' difficulty and see whether or not this affects the results when games are played.\\
It was therefore important to add some modularity to the question generation process and decide what parameters could be played with to influence the difficulty. Note that older events are obviously harder than recent events so we did not take into account the creation date of a post when discussing the difficulty of a question. We approached the problem from two angles, the selection of items and the generation of the question itself.\\
Some items will be harder to remember than others no matter the way the question is then generated. For instance, finding out precisely when something was posted is hard if it is in the middle of a lot of other posts because it was posted during a period of heavy usage of Facebook. On the other hand, isolated posts are really memorable because they are often about something so noteworthy that it made the user go on Facebook and talk about it when they were in the middle of a low usage period.\\
The way items are used when generating questions is also something that increases and decreases difficulty. The choice of other alternatives in a multiple choice questions has a lot of influence on the difficulty and the size of the proposed time periods on a time question has also an influence.

\section{Multiple Choice Questions}
Use of relationship between the users.
\section{Ordering Questions}
Reactions: standard deviation from the mean.
Pages: clustering: in same cluster -> hard, two clusters -> easy, cluster + non-cluster -> medium
\section{Geolocation Questions}
Cluster: in a large cluster -> easy, outside of cluster -> hard, in medium sized cluster -> medium
\section{Timeline questions}
Same as geolocation.
\section{Difficulty selection}