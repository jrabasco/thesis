\chapter{Game Design}
\section{Reminder}
\section{New Data}
As new services and new technologies enter the market of social medias, people keep changing the way they make use of said products. A lot of the platforms focus on giving more incentive to their users to expose a larger and larger part of their life on said services. They develop new ways for people to share their experiences and it gives social medias the ability to be a part of the everyday life of a lot of their users. Facebook is no exception to this, the more the people can share, the more they will use the platform and interact with their friends on it and the more likely they are to come back. To be able to exploit those resources as much as possible in ReminisceMe we must continuously be on the look for new ways we can update and refine our data model to improve our understanding of biographical memory.
\subsection{New reactions}
Since the 24\textsuperscript{th} of February 2016 \cite{reactrelease}, Facebook has introduced new ways their users can express their reactions to a post they see on their feed. These reactions complement the "Like" reaction which has been for years the only possible reaction. The five new reactions are "Love", "Haha", "Wow", "Sad" and "Angry". They represent new ways the people using the platform can express their feelings. It is important for this project to take those into account as they provide some nuances about how the different posts are perceived by the user. These new reactions are therefore now collected by the latest version of ReminisceMe.\\
To exploit this new source of information, we decided to add two sort of questions to the existing ones.
\paragraph{Who loved/laughed at/was impressed by/was sad about/was angry at this post you made?}
Following the same schema as "Who liked this post you made?", those are the most obvious choices. What we get from those is a bit of nuance, it helps showing if the player not only remembers someone they interacted with but the way the interaction happened.
\paragraph{Order the reactions on this post according to their number}
This question can be really trivial to answer in a lot of cases because in most cases a post generates mainly on type of reaction and the other are negligible. This happens because we tend to befriend people who mostly share the same opinions as we do and when we share something because we find it lovely, amusing, awesome, sad or revolting, the majority of people seeing it will share the same reaction as ourselves. However, in some instances, a post will generate a lot of polemic and attract the passions, resulting in a multitude of different reactions. In that instance, asking to order the reactions tells us if the user remembers about the discussion around that post, if the once seemingly hot topic is still in their mind and if they remember how their friend reacted to it.

Unfortunately, this feature being quite recent, the number of posts where these new reactions are used a lot is limited. The main reason is that for the large part of the existence of Facebook, these reactions did not exist and therefore the long time users are not used to it yet and simply did not produce a lot of data with these. This means that we might not be able to generate a lot of questions based on those new reactions and that we may therefore not be able to draw a lot of conclusions yet.
\subsection{Friends}
The application has been recently officially approved by Facebook. One of the perks of this approval, is that it gives access to more data. With those new access capabilities, we are now able to retrieve the friends of a user (plus or minus a couple which do not wish to be invited to use any application on Facebook and are therefore completely invisible to us). The friends of a user are an interesting piece of data as it can help us in the process of generating multiple choice questions. One of the recurring challenges when generating those questions is finding the wrong answers we display, we usually use people who interacted with the user in an other post. Some of the friends may never have interacted with the user on Facebook and therefore they would not appear in the names we gathered before. They are also better candidates when trying to build a multiple choice question: they are more likely to be known to the user (some non-friend liking a post once might go completely unnoticed and the user would easily know that this is not the answer) and can therefore help produce more difficult questions.\\
Another useful aspect of the friends is that they open the possibilities for new questions. One interesting property of a social network is that you often interact with a friend of a friend which you never actually add to your friends list, but they might appear from time to time on the publications you make that involve your common friend. Combining this aspect with the fact that heavy social networks users have large amounts of friends on the platform gives us a situation where some people might not be able to tell who their friends are. This is especially the case if the list of people presented to them is comprised of names of users who all reacted at some point to one of the player's publications. This question provides insight into a different kind of memory. While most of the other questions focus on how the user remembers events in their life, this one treats with the way they remember their relationships with others and the importance of friendship on a social media. <Needs result from pilot, if a lot of good answer then it is not that interesting, a lot of wrong might indicate something...>