\chapter{Game Design}
\section{Reminder}
\section{New Data}
A social network the size of Facebook can provide a lot of information that we can exploit to generate better questions. On top of that, the website is continuously improving and creating new kinds of content and users become more and more creative with this content. Continously updating our data model and getting more and more types of data from the social network is an important part of the development of ReminisceMe.
\subsection{New reactions}
Since the 24\textsuperscript{th} of February 2016 \cite{reactrelease}, Facebook has introduced new ways their users can express their reactions to a post they see on their feed. These reactions complement the "Like" reaction which has been for years the only possible reaction. The five new reactions are "Love", "Haha", "Wow", "Sad" and "Angry". These reactions represent new ways the people using the platform can express their feelings. It is important for this project to take those into account as they can improve our comprehension of memory by providing some nuances about how the different posts are perceived by the user. These new reactions are therefore now collected by the latest version of ReminisceMe and we created a new question especially for it (in addition to the obious ones we could generate following the same patterns as the "Like" reaction before): ordering the reactions on a post based on their number. This would tell us if people remember well or not if the impact of a certain post was mostly negative or positive. While being somewhat trivial in some obvious cases, the controversial posts made can be a bit more difficult to analyse and the answer might be influenced by how the person answering has evolved and grown in the years seperating them from the apparition of the post. Unfortunately, this feature being quite recent, it might be hard to draw conclusions from this type of question.
\subsection{Friends}
The application has been recently officially approved by Facebook. One of the perks of this approval, is that it gives access to more data. With those new access capabilities, we are now able to retrieve the friends of a user (plus or minus a couple which do not wish to be invited to use any application on Facebook and are therefore completely invisible to us). This piece of data is really interesting as it offers the opportunity to create new questions (ELABORATE MORE IF WE MANAGE TO CREATE ONE) and improve the previous questions by giving us more names to put as possible answers on some of the questions.