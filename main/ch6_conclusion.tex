\chapter{Conclusion}
During this iteration, we conducted two public pilots of Reminisce.me and managed to launch the first stable version of the mobile application. This is no small feat and it is worth to note that the chose technologies allowed for a rather fast development and produced a stable software infrastructure.\\\\
The engagement of our friends and family members to play the game was impressive, the people were more than eager to play the game to collect a lot of useful data and feedback. While a lot of small defects were brought up during these trials, the general feedback was mostly positive. People did enjoy play the game, remembering of old posts and sometimes being embarrassed by their old self. This is a really encouraging outcome, if we want this project to be successful, we need it to provide entertainment.\\\\
The result we obtained with the difficulty are not that good. It clearly seems that our implementation does not have a lot of effect on the actual difficulty of the questions. While we have anecdotal evidence that it is not completely inefficient (some user reported getting some really hard questions), the overall difficulty of the questions is still not completely mastered. It is not that encouraging but we have to take into account the fact that most of the people played less than 20 games and that a lot of them have a rather low Facebook activity. These factors alone do not explain the results\footnote{While doing the analytics, it showed that even trying to analyze only for users with high activity the results were about the same} it is still a sign that we need more data: more users who play more.\\\\
Even though the results are not as satisfying as we thought, we still manage to introduce the difficulty component to our question generation process. It may not be perfect yet but the infrastructure is there and the first step has been taken which is really positive result in itself.\\\\
In the feedback we got, it was often remarked that some of the posts people were presented with do not have a lot of content which makes them hard to remember (sometimes completely impossible, for instance when the post is just a few very generic words like "I'm done"\footnote{Actual example}). Most of the questions are good in that aspect, but the few bad ones add some fuzziness to the data. This is clearly an issue that would lead to a great improvement and a better understanding of the data we are dealing with if it were to be solved.\\\\
Reminisce.me has already proved that it could work and that it was enjoyable. It is now in a state where we can conduct multiple experiments and play with parameters easily. The future of this project lies in more measurements with more and more players generating as much data as possible. This would further our understanding of what makes questions difficult and help us generate better questions which are adapted in terms of difficulty to the person answering them. This would both help test some initial theories about autobiographical memory and improve the software at the same time.