\chapter{Conclusion}
During this iteration, we conducted two public pilots of Reminisce.me and managed to launch the first stable version of the mobile application. This is no small feat and it is worth to note that the chosen technologies allowed for a rather fast development and produced a stable software infrastructure.\\\\
The engagement of our friends and family members to play the game was impressive, the people were more than eager to play the game to collect a lot of useful data and feedback. While a lot of small defects were brought up during these trials, the general feedback was mostly positive. People did enjoy play the game, remembering of old posts and sometimes being embarrassed by their old self. This is a really encouraging outcome, if we want this project to be successful, we need it to provide entertainment.\\\\
We managed to implement a whole framework which is able to analyze the items we get from Facebook and predict how difficult it is to remember them based on a set of predefined criteria. This new version of the software also enables us to generate questions of varying difficulties and lays the foundation for the control and adaptation of the difficulty of our generated game boards. This infrastructure is an important step towards helping people understanding human memory, we can test the parameters we think are useful in the memorization process and see how it affects the performance of the users when playing the game.\\\\
The results obtained during the two trials show that this first version did not have all the expected impact but it does show that we can make this infrastructure work. We plan to further improve the board generation procedure by improving the diversity of generated questions. We also plan to do a lot more different trials in the future to help in th quest to discovering the secrets of autobiographical memory.\\\\
Besides continuously improving our understanding of memorability and difficulty, future work might need to also explore an orthogonal track: analyzing in more details the actual content of the posts. Instead of simply identifying the type of post (is it geolocalized? has it some image attached?) we could try to discover the actual subject of the post: what is this person talking about?. Not only would it help solving the issue mentioned before about posts with no clearly identifiable information, it could lead us explore a whole new range of parameters that might influence the memorability of data.\\\\
Reminisce.me has already proved that it could work and that it was enjoyable. It is now in a state where we can conduct multiple experiments and play with parameters easily. The future of this project lies in more measurements with more and more players generating as much data as possible. This would further our understanding of what makes questions difficult and help us generate better questions which are adapted in terms of difficulty to the person answering them. This would both help test some initial theories about autobiographical memory and improve the software at the same time.