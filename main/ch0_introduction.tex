\chapter*{Introduction}
\addcontentsline{toc}{chapter}{Introduction}
The advent of the Web reshaped the way in which humans memorize information, arguably unlike any other technological advance of the last decades. Rather than remembering the information itself, people are primed to find the needed information through Web search engines \cite{googlemem} . In light of this result, psychologists are starting to reconsider the standard framework for which memory is the process by which information is encoded, stored, and retrieved. Content that can be easily retrieved from the Web is encoded to enhance the recall for where to access it; therefore it is only partially stored. It is important to notice, though, that news articles, encyclopedic entries, video content, etc. are very different from the autobiographical memories that the users contribute daily to large Online Social Networks (OSNs). Our intuition is that the previous studies \cite{retenauto} \cite{howhappy} were not able to capture the idiosyncrasies of memory retention in social networks, because they were focused exclusively on very narrow aspects of our daily lives, rather than the variety embraced by OSNs.\\\\
Reminisce.me is a Web game that gathers data from the user’s Facebook activity. The game mechanics are rather simple: the user is presented with a variation of the famous Tic-Tac-Toe game, where to conquer a tile he is required to answer a few questions related to his own autobiographical memory. We designed 4 different types of questions to test information memorability on multiple
dimensions. In particular, we have :
\begin{itemize}
	\item Multiple choice questions, where the player has to remember who interacted with his posts, shared pictures, etc.
	\item Timeline-based questions, where the player has to remember when he shared a certain post, picture, etc.
	\item Order-based questions, where the player has to order items in temporal or popularity order (e.g., number of `likes' on Facebook)
	\item Geolocalisation questions, where the player has to remember where he was at the time he shared a certain post, picture, etc.
\end{itemize}
The project had already been presented to the public in June 2016. The data collected during this first pilot helped get more insight into people's memory about their activity on Facebook. The feedback received was really useful into finding what the next steps should be.\\
During this iteration, we wanted to explore the difficulty of the questions we are generating. The questions were previously generated on a purely random basis and no particular attention was given to generating multiple levels of difficulty. They were all designed in a way that we thought would make them fair and entertaining to answer. What makes a question difficult is inherently linked to what makes an event memorable, because an easy question is one for which the answer is easy to remember. The main goal was then to generate questions of varying difficulty based on our assumptions of what makes them easier or harder and try this by having people play the game. We created the architecture and did the groundwork to make it possible to make the difficulty of questions vary.\\\\
A second goal was also to improve the general enjoyability of the game as it is a really important aspect of the project, if we want people to use our application and answer questions we must make sure that their experience is as enjoyable as possible.\\\\
We decided to go with two public trials for this iteration, the first in the beginning of December 2016 to test the some of the improvements made and collect basic statistics about the people's performance when the questions with varying difficulty are not yet implemented. The second was held in early-mid January 2017 with all the difficulty related features implemented. More than 150 games were played by more than 40 different users during the second trial. The data gathered there was of great help to determine the efficiency of our implementation.\\
The following chapters outline the reasons behind some of the key decisions made during this iteration of the project and will show the resulting changes and effects on the application as well as the results of the two trials.