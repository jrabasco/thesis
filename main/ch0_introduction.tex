\chapter*{Introduction}
\addcontentsline{toc}{chapter}{Introduction}
The advent of the Web reshaped the way in which humans memorise information, arguably
unlike any other technological advance of the last decades. Rather than remembering the
information itself, people are primed to find the needed information through Web search
engines \cite{googlemem} . At the light of this result, psychologists are starting to reconsider the standard
framework for which memory is the process by which information is encoded, stored, and
retrieved. Content that can be easily retrieved from the Web is encoded to enhance the recall
for where to access it; therefore it is only partially stored.
It is important to notice, though, that news articles, encyclopedic entries, video content, etc.
are very different from the autobiographical memories that the users contribute daily to large
Online Social Networks (OSNs). Our intuition is that the previous studies [2] [3] were not
able to capture the idiosyncrasies of memory retention in social networks, because they were
focused exclusively on very narrow aspects of our daily lives, rather than the variety embraced
by OSNs.
Reminisce.me is a Web game that gathers data from the user’s Facebook activity. The game
mechanics are rather simple: the user is presented with a variation of the famous Tic-Tac-
Toe game, where to conquer a tile he is required to answer few questions related to his own
autobiographical memory.
We designed 4 different types of questions to test information memorability on multiple
dimensions. In particular, we have :
\begin{itemize}
	\item Multiple choice questions, where the player has to remember who interacted with his posts, shared pictures, etc.
	\item Timeline-based questions, where the player has to remember when he shared a certain post, picture, etc.
	\item Order-based questions, where the player has to order items in temporal or popularity order (e.g., number of « likes » on Facebook)
	\item Geolocalisation questions, where the player has to remember where he was at the time he shared a certain post, picture, etc.
\end{itemize}
The project started with designing the user experience and how the questions are presented
to the user. Multiple ui elements have been drafted until we reached a state where we could
present the rich and diverse content from Facebook and still have an intuitive design.
We than had to decide on the architecture and the technologies we wanted to use. One
important part was to build scalable components which would be capable to handle peak
requests if the game becomes viral and successful. We also wanted to use battle tested
frameworks where we could be assured, that even after a few years the support and continuous
development would not stop due to the lack of success or money of the providers. We also had
to consider the finical impact and avoid any services which might be free in the beginning but
would ramp up in cost as soon as Reminisce.me becomes successful.
We then started to implement the different components with the architecture chosen and once
we had a stable version, we submitted it to Facebook. As soon as we received the approval we
launched a first pilot and invited friends and family to test the application and to collect some
data.
After only 5 days users answered more than 1’000 questions and played more than 4 games on
average. Through the data collected during the pilot we could create some initial results and
draw conclusions on the performance of the players based on the question type.
In the following chapters we outline each of the phases in detail and substantiate why these
decisions have been taken.
