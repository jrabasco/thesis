\chapter{Continous improvement}
\section{New data}
A social network the size of Facebook can provide a lot of information that we can exploit to generate better questions. On top of that, the website is continuously improving and creating new kinds of content and users become more and more creative with this content. Continously updating our data model and getting more and more types of data from the social network is an important part of the development of ReminisceMe.
\subsection{New reactions}
Since the 24\textsuperscript{th} of February 2016 \cite{reactrelease}, Facebook has introduced new ways their users can express their reactions to a post they see on their feed. These reactions complement the "Like" reaction which has been for years the only possible reaction. The five new reactions are "Love", "Haha", "Wow", "Sad" and "Angry". These reactions represent new ways the people using the platform can express their feelings. It is important for this project to take those into account as they can improve our comprehension of memory by providing some nuances about how the different posts are perceived by the user. These new reactions are therefore now collected by the latest version of ReminisceMe and we created a new question especially for it (in addition to the obious ones we could generate following the same patterns as the "Like" reaction before): ordering the reactions on a post based on their number. This would tell us if people remember well or not if the impact of a certain post was mostly negative or positive. While being somewhat trivial in some obvious cases, the controversial posts made can be a bit more difficult to analyse and the answer might be influenced by how the person answering has evolved and grown in the years seperating them from the apparition of the post. Unfortunately, this feature being quite recent, it might be hard to draw conclusions from this type of question.
\subsection{Friends}
The application has been recently officially approved by Facebook. One of the perks of this approval, is that it gives access to more data. With those new access capabilities, we are now able to retrieve the friends of a user (plus or minus a couple which do not wish to be invited to use any application on Facebook and are therefore completely invisible to us). This piece of data is really interesting as it offers the opportunity to create new questions (ELABORATE MORE IF WE MANAGE TO CREATE ONE) and improve the previous questions by giving us more names to put as possible answers on some of the questions.
\section{Comfort of the user}
One of the main goals of ReminisceMe is to get as many information from the players. It is crucial that the users play as many games as possible and that they keep playing for as long as possible. The more a person plays the more detailed the analysis can be. It is also important that the users enjoy the game enough for them to talk about it and attract more users. For that to be possible, we need the game to be enjoyable and the player to feel comfortable while playing. A lot of measures have been been put in place to make the whole experiment as good as possible.
\subsection{Blacklist}
While the application was being tested by different users, some users came across content related to people they did not wish to hear about. While this is a really unfortunate situation and it might diminish our capacity to generate questions, we felt compelled to add the possibility for the user to ignore users they do not want any mention of on their game boards. The main reason behind this decision is that, depending on the relationship between a potential player and some of the people mentioned in the questions, the discomfort can be big enough for the user to not want to play the game again. Unfortunately, it is impossible for us to predict when such a situation might arise and we cannot do anything preemptively [HIGHLIGHT BLACKLIST IN INTERFACE?], but we hope that having the ability to ignore people is sufficiant to make users stay after encountering such a situation.
\subsection{Mobile improvements}
The biggest part of Facebook users are mobile users \cite{mobileusage}. Therefore, we expect a decent portion of our user base to be mobile users, playing while on transportation or during a break. Making the experience great for these players is then an important concern.
\subsubsection{Mobile application}
The main challenges to have a smooth mobile experience are design issues and bandwidth limitations. One of the best ways to overcome these difficulties is to provide the user with a standalone mobile application. This both helps with serving a nice responsive mobile layout and allows the user to download the application before using it, this is helpful as the desktop version is quite heavy and is not really friendly with the narrow badnwidth mobile data offers.
\subsubsection{No post ordering}
Despite having a responsive design on a mobile application, certain elements cannot be displayed properly and even look weird on a large screen. The main difficulty about displaying posts is that their length and content cannot predicted and some large posts (which are consequently also more interesting) cannot be rendered nicely inside an ordering item. The decision was therefore made to drop all the questions which would ask to reorder posts.
\subsection{Privacy and security}
We are asking our users to share private and potentially intimate information with us. We have to take extreme care when manipulating the data we get and when serving the questions to the user. The project architecture was improved so that the different containers do not communicate with the outside world directly but instead goes through an NginX\footnote{\url{https://www.nginx.com/}} proxy. This helped us with the setup of a secure communication between the application and our servers using the latest and most up to date security protocols\footnote{\url{https://www.ssllabs.com/ssltest/analyze.html?d=reminisce.me&latest}}. We also need the users to trust that we are using a secure connection so we obtained a security certificate issued by Let's Encrypt\footnote{\url{https://letsencrypt.org/}} which will be recognized by any modern browser (screenshot of the green lock?) as safe.\\
The above-mentioned proxy server and the ability we get from Facebook to identify the application's administrators helped us hide some parts of the application to the general public. In particular, only the administrators can access the newly-created log page and the feed back page (both potentially contain private information).