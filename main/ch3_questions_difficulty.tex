\chapter{Questions difficulty}
The first launch of Reminisce.me to the public helped a lot in understanding better how people would perform while answering the questions we have in the application. It was determined that on average the people participating in the pilot managed to answer more accurately than the random baseline. It was however hard to draw more really precise and advanced conclusions. There are two main reasons which impaired the ability to draw conclusions. THe first reason is the number of generated questions which is not always sufficient to draw conclusions, which was the case especially for the geolocation questions. The second reason is that the difficulty of the questions was generated randomly and the lack of ability to change the difficulty made it difficult to adjust it for each player or even try to have the same amount of difficulty on each type of question. Having a measure of difficulty for each type of question would help understanding better what makes something hard to remember besides having more possible answers\footnote{Note that for some cases like the Geolocation questions, having a large number of possible answers does not make it necessarily extremely hard. For this particular case, the set of possible answers is really large but the gathered data points are often clustered around particular areas which make it easier to answer.} or being really far in the past. While we cannot do anything about the number of possible questions (some types of data are more rare than others and we can do little about it), it is possible to have influence on the difficulty.

\section{Multiple Choice Questions}
\section{Ordering Questions}
\section{Geolocation Questions}
\section{Timeline questions}